\chapter{Research Plan} \label{chap:ResearchPlan}




\section{Theme} \label{sec::Theme}

    Correlação de noticias com variaveis de mercado utilizando NLP.

\subsection{Theme Delimitation} \label{subsec::ThemeDelimitation}

   Análise sentimental de postagens em redes sociais relacionadas a agro utilizando NLP para compreender a situação atual que uma determinada região e ou grupo de pessoas estão passando, e como seria possível melhor servir as mesmas.
    
    A utilização de APIs de redes sociais tais como Twitter, Facebook, YouTube, Instagram pode permitir a coleta de dados sobre as opiniões que os seus usuários estão expondo em tempo real. A partir destes dados, a análise utilizando técnicas de NLP pode permitir a retirada de informações que demonstrem de maneira mais simples e completa a situação dos mesmos.

\lipsum[2-3]

\section{General Objective} \label{sec:objective}


\subsection{Specific Objectives}
\begin{enumerate}
    \item aaaaaaaaaaaaaaa
    \item bbbbbbbbbbbbb
    \item CCCCCCCCCC
    \item DDDDDDDDDDD
    \item FFFFFFFFFFFF
    \item GGGGGGG
\end{enumerate}


\section{Justification}\label{sec:justification}



\section{Problem} \label{sec::Problem}



\section{Hypothesis} \label{sec::Hypothesis}
\begin{enumerate}
    \item A is equal to C
    \item D is bigger than G
\end{enumerate}


\section{Methodology} \label{sec:Methodology}

\subsection{Approach}

\subsection{Procedures}

\subsection{Tecniques}

\subsection{Hyphoteses Validation}


\section{Budget} \label{sec:budget}

%table example

\section{Schedule of Activities} \label{sec:schedule_activities_table}

%table example